\documentclass[10pt]{cv}
\usepackage{natbib}
\usepackage{etaremune}
\usepackage{fancyhdr}
\usepackage{url}
\usepackage{hyperref}

\font\cap=cmcsc10
 
\topmargin 0pt
\headheight 0pt
\headsep 0pt
\textheight 9in
\parindent 0pt
\parskip \baselineskip
\topmargin 0in
\oddsidemargin -0.25in
\evensidemargin -0.25in

\textwidth 6.9in

\setlength{\topmargin}{-0.25in}
\setlength{\headheight}{0in}
\setlength{\headsep}{0in}
\setlength{\topskip}{0.4in}
\setlength{\textheight}{9.3in}

\pagestyle{fancy}
\fancyhf{}
\renewcommand{\headrulewidth}{0pt}
\lfoot{Rocio Ayelen Kiman}
\cfoot{\thepage}
\rfoot{Curriculum Vitae}
\newcommand\tab[1][1cm]{\hspace*{#1}}

\hypersetup{colorlinks = true} %comment out for any printed version! 

%footer options:
\setcounter{page}{1} %set the number in the second set of curly braces to the number that would be on the first page
\newcommand{\secondfooter}{Publications} %the footer you'd like on the publications half should be uncommented
%\newcommand{\secondfooter}{Curriculum Vitae} 
\newcommand{\firstpagestyle}{empty} %the style of the first page: fancy = include footer, empty = don't.
%\newcommand{\firstpagestyle}{fancy}

\begin{document}
\thispagestyle{\firstpagestyle} 
\begin{center}
{\LARGE \textbf{\sc Rocio Ayel\'en Kiman}}\\
\bigskip
\bigskip
%\smallskip
\end{center}
\normalsize
  
\addresses
{
Citizenship: Argentina\\
Email: rociokiman@gmail.com\\
Phone: +1 908 7372957 \\
}
{
The Graduate Center, CUNY\\
Office Address: American Natural History Museum\\
Central Park West at 79th Street\\
New York, NY 10024, USA\\ 
\\
}

\begin{llist}

\sectiontitle{Education}
The Graduate Center, CUNY, USA
\location{2016 - Expected Spring 2021}
Ph.D. in Physics\\
Thesis Advisor: Prof. Kelle Cruz

Universidad de Buenos Aires, Argentina
\location{2011 - 2016}
Licenciatura in Physics\\
Thesis Title: ``Higgs boson pair production at the LHC''\\
Thesis Advisor: Prof. Daniel de Florian

\sectiontitle{Appointments}

Graduate Research Assistant
\location{August 2017 - Present}
Research Fundation, CUNY

Non-Teaching Adjunct
\location{August 2017 - Present}
Hunter College, CUNY

Non-Teaching Adjunct
\location{August 2016 - August 2017}
Graduate Center, CUNY

Teaching Assistant
\location{March 2016 - August 2016}
Universidad de Buenos Aires, Argentina\\
Teaching assistant for Classical Mechanics classes.


\sectiontitle{Grants \&\\ Awards}
Doctoral Student Research Grant (Round 14) for $\$875$, 2019.

Provost’s Pre-Dissertation Research Fellowship for the Sciences for $\$5,000$, 2019.

Cycle 49 PSC-CUNY Research Award for $\$6,000$ with Prof. Kelle Cruz, 2018.

CUNY Science Scholarship, 2016.


\sectiontitle{Publications \\\& Tutorials}
\textit{Exploring the age dependent properties of M and L dwarfs using \textit{Gaia} and SDSS.} Kiman, R., Schmidt, S.J., Angus, R., Cruz, K.L., Faherty, J.K. \& Rice, E. Submitted to Astronomical Journal, currently under revision after review.

\href{https://github.com/rkiman/astropy-tutorials/blob/user-defined-fitter/tutorials/notebooks/User-Defined-Fitter/User-Defined-Fitter.ipynb}{User-Defined-Model.} Astropy Tutorial.

\href{https://github.com/rkiman/astropy-tutorials/blob/user-defined-fitter/tutorials/notebooks/Models-Quick-Fit/Models-Quick-Fit.ipynb}{Models-Quick-Fit.} Astropy Tutorial.

\sectiontitle{Talks}

\textit{Finding age relations for low mass stars using magnetic activity and kinematics.} Contributed talk, Big Apple Magnetic Fields Conference, January 24-25, 2019, Center for Computational Astrophysics at the Flatiron Institute, NY, New York, USA. 

\textit{Age Dating Low Mass Stars Using Galactic Kinematics.} Contributed talk, Cool Stars, July 30 to August 3, 2018, Boston-Cambridge, USA.

\textit{Age Dating Low Mass Stars Using Galactic Kinematics.} Contributed talk, Graduate Research Conference, May 10, 2018, College of Staten Island, NY, USA.

\textit{Gaia-Cupid: Age-dating low mass stars using galactic kinematics.} Lightning Talk, SDSS-IV Collaboration Meeting, 24-26 July 2017, Santiago, Chile. 

\sectiontitle{Poster\\Presentations}

\textit{Finding age relations for low mass stars using magnetic activity and kinematics.} Kiman, R., Schmidt, S.J., Angus, R., Cruz, K.L., Faherty, J.K. \& Rice, E., AAS 233, 6-10 January, 2019, Seattle, Washington, USA.

\textit{Age Dating Low Mass Stars Using Galactic Kinematics.} Kiman, R., Schmidt, S.J., Angus, R., Cruz, K.L., Faherty, J.K. \& Rice, E., Cool Stars, July 30 to August 3, 2018, Boston-Cambridge, USA.

\textit{Age Dating Low Mass Stars Using Galactic Kinematics.} Kiman, R., Cruz, K.L., Angus, R., Schmidt, S.J. \& Faherty, J.K, AAS 231, 8-12 January, 2018, Washington DC, USA. 

\textit{Gaia-Cupid: Age-dating low mass stars using galactic kinematics.} Kiman, R., Cruz, K.L, Angus, R., Schmidt, S.J \& Faherty, J.K., BDExoCon II, 26-27 October 2017, Delaware, USA. 

\textit{Photolysis of caged compounds with controlled temporal modulation.} Kiman, R., Camino, P., Ponce Dawson, S., Lopez, L., Piegari, S., 99º RNF-AFA (National Meeting of the Physical Association Argentina) 22-25 September, 2014, Buenos Aires, Argentina.


\sectiontitle{Conferences\\Attended}
\textit{.Astronomy X.} 24-27 September 2018, Baltimore MD, USA.

\textit{Gaia Sprint.} 4-8 to June, 2018, Center for Computational Astrophysics at the Flatiron Institute, NY, New York, USA.

\textit{Python in Astronomy.}  April 30 to May 4, 2018, Center for Computational Astrophysics at the Flatiron Institute, NY, New York, USA. 

\textit{Gaia DR2 Sprint.} 25-27 April, 2018, Center for Computational Astrophysics at the Flatiron Institute, NY, New York, USA. 

\sectiontitle{Schools\\Attended}

\textit{IYAS on the scientific exploration of the Gaia data.} February 26 to March 2, 2018, Paris, France.

\textit{La Serena School of Data Science.} 21-29 August, 2017, La Serena, Chile.

\sectiontitle{Outreach\\Activities}
Outreach Assistant
\location{2014 - 2016}
Universidad de Buenos Aires, Argentina\\
\tab Presenter at the ``Physics week'' for high-school students, 2014-2015.\\
\tab Presenter at the ``­Museum's night'', 2014-2015.\\
\tab Presenter at the Book Fair in Buenos Aires, May 2015.\\
\tab Monthly outreach talks for high-school students about the career in Physics.


\end{llist}
\end{document}